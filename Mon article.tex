\documentclass[12pt,a4paper,titlepage]{article}

\usepackage[utf8]{inputenc}
\usepackage[T1]{fontenc}
\usepackage[francais]{babel}
\usepackage{datetime}  


\usepackage[letterpaper,top=2cm,bottom=2cm,left=3cm,right=3cm,marginparwidth=1.75cm]{geometry}

% Useful packages
\usepackage{amsmath}
\usepackage{graphicx}
\usepackage[colorlinks=true, allcolors=blue]{hyperref}

\title{Architecture des Systèmes d'Information {}}
\author{Moussa Kourouma\\
   Matricule : 000529581\\
   Université Libre de Bruxelles\\
   Master en Sciences et Technologie de l'Information et de la Communication\\
   Bruxelles - Belgique\\
   \texttt{moussa.kourouma2@ulb.be}}
\date{\today}

\maketitle

\begin{document}
\maketitle

\clearpage
\vspace*{\fill}
\begin{center}
\begin{minipage}{.6\textwidth}
\begin{center}
\huge {\textbf {L’Impact des réseaux sociaux sur les jeunes.}} 
\end{center}
\end{minipage}
\end{center}
\vfill % equivalent to \vspace{\fill}
\clearpage

\tableofcontents


\newpage

\section{Introduction}
La notion des réseaux sociaux ont vu jour aux Etats-Unis en
1995avec l’avènement des premiers réseaux apparu sur
internet par Randy, appelé «classmates » qui à été utilisé
par tous les continents en 2004. Les réseaux sociaux se sont
développés à début du XXIème siècle suite à l’avènement de
la nouvelle technologie de l’information et de
communication.
Nombreux connaissent aujourd’hui les outils utilisés comme
facebook, twitter, google, snapchat, instagram pour ne cité
que ceux-ci, ses outils sont utilisés partout dans le monde,
rare sont ce qui ne le connaissent pas.
Les réseaux sociaux occupe une grande place de nos activités,
c&#39;est-à-dire nous vivons dans un monde ou tout est numérisé,
être connecté sur les différents outils comme facebook qui
est une application dans la quelle il faut avoir un profil pour

être identifier partout dans le monde pour être en contacte
avec nombreuses personne dans le monde. Facebook est un
compte tu peux retrouver des anciens amis, beaucoup
d’évenement se passe sur cette application comme suivre les
actualités de n’importe quel pays, on peut regarder des clips
vidéos et même regarder les films. Facebook aide surtout les
étudiants de la Belgique de trouver un travail, et des stages
pour les étudiants sur la plateforme « jobétudiants »
L’impacte des reseaux sociaux touche beaucoup à l’utilisateur
qui laisse toutes leur activité pour etre sur les reseaux.
D’ailleurs ces outils facilitent des rencontres qu’on peut
qualifié d’extraordinaire, la preuve en est que les sites de
rencontres qui met les individus en relation de façon
phénoménale. Par contre ses outils vont impactés nos
manière de travail. On peut s’en passer des réseaux sociaux il
est indispensable dans la vie de chacun d’entre nous, ils
occupent une place très importante dans la société, publier
une nouvelle sur facebook ou twitter peut faire le buzz à
l’échelle internationale quand il tombe dans la main d’une
personne mal intentionné.
L’utilisation des réseaux sociaux sont devenir aujourd’hui très
indispensable dans la vie des jeunes adolescents. Une
information sur les réseaux comme : twitter peut entraver un
buzz sur toutes les applications comme twitter, facebook,
instagram. Information peut se développer à l’échelle
nationale ou internationale, quand une information est publié
toutes les internautes reçoivent à la minute qui suivre, ils ont
un certain pouvoir de fédérer, ressembler les informations et
créer une sorte de conversation dans le monde. Ces
publication peuvent parfais tomber dans la main des

mauvaises personnes, en l’occurrence des jeunes. Les réseaux
sociaux peuvent aussi causer la dépression, regard critique
sur les gens, les réseaux sociaux peuvent être la cause de
malaise pour les jeunes.

\section{Définition :}
Les réseaux sociaux est un ensemble d’outil qui permet aux
gens de ce connecté, faire rencontre sur des outils comme
Messenger, Whatsapp, Facebook, Snapchat, instagram pour
ne citer que ceux-ci.
\section{Comment les réseaux sociaux ont envahi le cœur des jeunes
adolescents pour le meilleur et pour le pire?}
Entre 2011 et 2013, Le psychologue Américain Jonathan
Haidt a remarqué une forte augmentation d’une maladie de
dépression chez les jeunes adolescents en Amérique sous
l’effet de l’utilisation des réseaux sociaux par les jeunes. Cette
utilisation des réseaux sociaux a causer un certain nombre de
jeunes admis à l’hôpital pour une dépression mental causer
par la forte utilisation des réseaux sociaux. Sur 100.000
jeunes âgées de 15 à 19 ans, 62% des jeunes filles utilisatrice
des réseaux sociaux ont été admises à l’hôpital. En plus, Entre
1999-2019, un certain nombre de cas de suicides des jeunes
adolescents ont été enregistré sur 1 million des jeunes filles.
Plus de 151% des jeunes filles se suicides. En 2009 ce constat
fait par les médecins qu’il y a une augmentation de suicide.
En effet, en été 2006 une date très importante dans l’histoire
des réseaux sociaux, la plus par des réseaux son devenus
disponible sur les téléphones, et a commencé à prendre de
l’ampleur des années suivantes. L’arriver du covid qui à

déclenché le confinement, une augmentation de l’utilisation
d’internet a été enregistré chez les jeunes fille japonaises.
Une association tokyoïte prévient sur la hausse des suicides
des jeunes adolescentes. La plus majeure partie des jeunes
adolescents dépendent des réseaux sociaux. 


\end{document}
