\documentclass[12pt,a4paper,titlepage]{article}

\usepackage[utf8]{inputenc}
\usepackage[T1]{fontenc}
\usepackage[francais]{babel}
\usepackage{datetime}  


\usepackage[letterpaper,top=2cm,bottom=2cm,left=3cm,right=3cm,marginparwidth=1.75cm]{geometry}

% Useful packages
\usepackage{amsmath}
\usepackage{graphicx}
\usepackage[colorlinks=true, allcolors=blue]{hyperref}

\title{Architecture des Systèmes d'Information {}}
\author{Moussa Kourouma\\
   Matricule : 000529581\\
   Université Libre de Bruxelles\\
   Master en Sciences et Technologie de l'Information et de la Communication\\
   Bruxelles - Belgique\\
   \texttt{moussa.kourouma2@ulb.be}}
\date{\today}

\maketitle

\begin{document}
\maketitle

\clearpage
\vspace*{\fill}
\begin{center}
\begin{minipage}{.6\textwidth}
\begin{center}
\huge {\textbf {L’Impact des réseaux sociaux sur les jeunes.}} 
\end{center}
\end{minipage}
\end{center}
\vfill % equivalent to \vspace{\fill}
\clearpage

\tableofcontents


\newpage

\section{Introduction}
La notion des réseaux sociaux ont vu jour aux Etats-Unis en
1995avec l’avènement des premiers réseaux apparu sur
internet par Randy, appelé «classmates » qui à été utilisé
par tous les continents en 2004. Les réseaux sociaux se sont
développés à début du XXIème siècle suite à l’avènement de
la nouvelle technologie de l’information et de
communication.
Nombreux connaissent aujourd’hui les outils utilisés comme
facebook, twitter, google, snapchat, instagram pour ne cité
que ceux-ci, ses outils sont utilisés partout dans le monde,
rare sont ce qui ne le connaissent pas.
Les réseaux sociaux occupe une grande place de nos activités,
c&#39;est-à-dire nous vivons dans un monde ou tout est numérisé,
être connecté sur les différents outils comme facebook qui
est une application dans la quelle il faut avoir un profil pour

être identifier partout dans le monde pour être en contacte
avec nombreuses personne dans le monde. Facebook est un
compte tu peux retrouver des anciens amis, beaucoup
d’évenement se passe sur cette application comme suivre les
actualités de n’importe quel pays, on peut regarder des clips
vidéos et même regarder les films. Facebook aide surtout les
étudiants de la Belgique de trouver un travail, et des stages
pour les étudiants sur la plateforme « jobétudiants »
L’impacte des reseaux sociaux touche beaucoup à l’utilisateur
qui laisse toutes leur activité pour etre sur les reseaux.
D’ailleurs ces outils facilitent des rencontres qu’on peut
qualifié d’extraordinaire, la preuve en est que les sites de
rencontres qui met les individus en relation de façon
phénoménale. Par contre ses outils vont impactés nos
manière de travail. On peut s’en passer des réseaux sociaux il
est indispensable dans la vie de chacun d’entre nous, ils
occupent une place très importante dans la société, publier
une nouvelle sur facebook ou twitter peut faire le buzz à
l’échelle internationale quand il tombe dans la main d’une
personne mal intentionné.
L’utilisation des réseaux sociaux sont devenir aujourd’hui très
indispensable dans la vie des jeunes adolescents. Une
information sur les réseaux comme : twitter peut entraver un
buzz sur toutes les applications comme twitter, facebook,
instagram. Information peut se développer à l’échelle
nationale ou internationale, quand une information est publié
toutes les internautes reçoivent à la minute qui suivre, ils ont
un certain pouvoir de fédérer, ressembler les informations et
créer une sorte de conversation dans le monde. Ces
publication peuvent parfais tomber dans la main des

mauvaises personnes, en l’occurrence des jeunes. Les réseaux
sociaux peuvent aussi causer la dépression, regard critique
sur les gens, les réseaux sociaux peuvent être la cause de
malaise pour les jeunes.

\section{Définition :}
Les réseaux sociaux est un ensemble d’outil qui permet aux
gens de ce connecté, faire rencontre sur des outils comme
Messenger, Whatsapp, Facebook, Snapchat, instagram pour
ne citer que ceux-ci.
\section{Comment les réseaux sociaux ont envahi le cœur des jeunes
adolescents pour le meilleur et pour le pire?}
Entre 2011 et 2013, Le psychologue Américain Jonathan
Haidt a remarqué une forte augmentation d’une maladie de
dépression chez les jeunes adolescents en Amérique sous
l’effet de l’utilisation des réseaux sociaux par les jeunes. Cette
utilisation des réseaux sociaux a causer un certain nombre de
jeunes admis à l’hôpital pour une dépression mental causer
par la forte utilisation des réseaux sociaux. Sur 100.000
jeunes âgées de 15 à 19 ans, 62% des jeunes filles utilisatrice
des réseaux sociaux ont été admises à l’hôpital. En plus, Entre
1999-2019, un certain nombre de cas de suicides des jeunes
adolescents ont été enregistré sur 1 million des jeunes filles.
Plus de 151% des jeunes filles se suicides. En 2009 ce constat
fait par les médecins qu’il y a une augmentation de suicide.
En effet, en été 2006 une date très importante dans l’histoire
des réseaux sociaux, la plus par des réseaux son devenus
disponible sur les téléphones, et a commencé à prendre de
l’ampleur des années suivantes. L’arriver du covid qui à

déclenché le confinement, une augmentation de l’utilisation
d’internet a été enregistré chez les jeunes fille japonaises.
Une association tokyoïte prévient sur la hausse des suicides
des jeunes adolescentes. La plus majeure partie des jeunes
adolescents dépendent des réseaux sociaux. 

\section{Conséquences des réseaux sociaux :}
Les réseaux sociaux ont une conséquence désastreuse qui
touche le plus souvent la vie des jeunes filles.
En effet, sur les réseaux sociaux, certaine personne
présentent leur mode de vie, expose leur corps. A cause de
tout ce comportement sur les réseaux sociaux, les jeunes
filles pratiquent la même chose qui pousse à entrainer une
perte d’estime de soi.
Les réseaux sociaux sont devenus plus à la mode en Amérique
du Nord plus que chez les français. Beaucoup de gens
utilisent ces réseaux pour intimider les internautes, ils
utilisent ces réseaux pour harceler les gens en rédigeant des
messages menassent, des photos qui leur pousse à
l’excitation. Cette histoire est plus réel, ces personnes mal
intentionné qui se déguisent en en fille sur les réseaux pour
avoir des vidéos des jeunes hommes. Beaucoup de
pédophiles s’en servent des réseaux sociaux pour séduire et
convaincre les fillettes et les jeunes adolescentes pour
obtenir les photos nues des fillettes. La majeure partie des
utilisateurs des réseaux sociaux son les jeunes, ces personnes
qui se déguisent en fille savent ou trouver ces jeunes sur les
réseaux sociaux. C’est pour cette raison, il est important de
contrôler les jeunes sur les réseaux sociaux.

Pour eux il n’y a pas assez de différence entre la vie de
professionnelle et la vie privé, les utilisateurs n’ont plus de
limite entre leur vie privée et leur vie professionnelle.
Les arnaqueurs informatiques sont devenus de plus en plus
très fort dans le vol d’identité via les réseaux sociaux.
Beaucoup de gens se servent des réseaux sociaux pour avoir
accès des renseignements personnels et privé.
Certains se servent pour rentrer dans l’ordinateur des gens
pour voler des mots de passe des cartes crédit, carte
bancaire, numéro de l’assurance sociale et plus encore.
D’une part les réseaux sociaux sont des sources
d’information, et d’autre par ils sont aussi à la base des
désinformations. Ces information sont diffusés à la minute ou
elle est partager que cela soit vrai ou fausse. C’est pour cette
raison ceux qui utilisent les réseaux sociaux doivent d’abord
s’informer être sur de l’information avant de partager.
On constate que ceux qui utilisent les réseaux sociaux, en
particulier sont les jeunes adolescents qui se comparent
toujours à des images publiés par les gens. Les réseaux
sociaux ont en particulier très bien des cotés très dangereux,
pourtant son conçu pour se faire des amis et des rencontres.
Ces jeunes adolescents ne veulent absolument rien manquer,
pour ce fait ils ont toujours leurs téléphones en main pour ne
rien manquer sur les réseaux sociaux… pendant le moment ils
sont connectés ils oublient même leur existence, ils sont
dominés par les publications, d’autre sentent leur existence
dans le monde virtuel.
Pourtant, c’est vrai que les réseaux sociaux peuvent être
parfois agréable et satisfaisant, l’utilisation abusée de ces
réseaux peuvent également entrainer une souffrance
mentale du point de vue psychologique comme exemple, la

personne peut être accro, qui peut entrainer ; la dépression
ou d’irritabilité, pendant que la personne est connecté il se
sent dans un monde virtuel, en ce moment d’utilisation il
peut avoir des émotions très désagréable telle la honte.
En bref, le fait qu’ils n’arrivent pas à diminuer l’utilisation de
des réseaux sociaux ou encore le fait de passer assez de
temps sur la connexion peut être à l’origine de la solitude, par
exemple la personne n’aura pas de temps d’être avec ces
amis ou même sortir de chez soi, il sera tellement dominer
par l’internet qu’il n’aura pas le temps pour autre activités,
cela peut même causer des pertes d’emploi.
Le fait de passer plusieurs minutes et heures sur les réseaux
sociaux déjà montre à quel point c’est une perte de temps,
l’utilisation abusif d’internet rend de nos jour les utilisateur
très paresseux, sa force est de rompre la solitude cela pousse
les utilisateurs de s’isoler, rester connecter durant toute la
journée entrain d’échanger avec les gens à travers son écran
plutôt qu’être en contact avec les dans le monde réel, alors
que le contact avec les gens dans le monde réel va
complément disparaitre. Cette solitude peut causer ce
comportement blizzard vis-à-vis des gens du monde réel.

\section{Cyberdépendance : }
Nous constatons que la cyberdépendance peut ’être la cause
de plusieurs effet négatif sur les réseaux et dans la vie de la
personne. Pour d’autres, être conscient des problèmes ou
danger de ce que qui se passe sur les réseaux ne suffit pas de
régler le problème. Pour d’autres, quand l’utilisation des

réseaux sociaux sont exagérés par les individus, ils sont
attendent d’une maladie mentale alors suivre un certain
traitement psychologique seront un bon traitement pour
l’individu. Alors dans ce cas, nous constatons l’intervention de
deux temps à savoir :
Observation : observation de soi même sur l’internet est un
bon signe pour l’individus, le fait de d’observer ces
publications, ses comportements est un bon départ qui peut
permettre à l’individus de faire une prise de conscience qui va
lui permettre de bien se comporter et de bien intervenir par
la suite.
Agir : en 1999, Young fait une proposition de série dans la
quelle donne les conseils à tous les utilisateurs des réseaux à
utiliser l’internet, mais de la façon approprié et saine. Dans
cette série de conseil nous suggère également de bien
pratiquer cette activité régulièrement et des comportements
souvent opposé. Par exemple, faire ces exercices, prendre sa
douche et prendre son déjeuner avant de se connecter,
toutes personne qui le fait habituellement ont moins de
problème pour l’utilisation des réseaux sociaux.

\end{document}
